% !Mode:: "TeX:UTF-8"
\documentclass[UTF8,10pt,a4paper]{ctexart}

\usepackage{graphicx}
\usepackage{CJK}
%\usepackage{url}
\usepackage{fancyhdr}
\begin{document}

\title{毕业论文}
\date{}

\author{张宇}

\maketitle

\tableofcontents
\thispagestyle{fancy}
\lhead{} % 页眉左,需要东西的话就在{}内添加
\chead{} % 页眉中
\rhead{} % 页眉右
\lfoot{} % 页脚左
\cfoot{\thepage} % 页脚中
\rfoot{} %页脚右,\thepage 表示当前页码
\renewcommand{\headrulewidth}{0pt} %改为0pt即可去掉页眉下面的横线
\renewcommand{\footrulewidth}{0.5pt} %改为0pt即可去掉页脚上面的横线
\pagestyle{fancy}
%\rfoot{\thepage}

\newpage

\section{引言}

  \subsection{项目背景}
  中国现阶段随着教育体制的改革,竞赛成为影响中学生升学的重要手段,其中与本专业计算机相关的一个竞赛是全国青少年信息学奥林匹克联赛,
  而为了让中学生能够更多接触学习这类知识,各地中学都在安排师资力量,以辅导班的形式对学生进行信息学竞赛辅导,
  以期望他们能够在竞赛中取得比较好的成绩,在之后的高考选拔中也就占有了一定优势。

  作为一项全国性的信息学奥林匹克赛事,联赛面向初中和高中阶段的在校学生,目的
  是为了给这些中学阶段的学生一个学习新知识的窗口,向他们普及计算机方面的知识,激发他们的兴趣,也为了培养更多这方面的人才,
  另外,学校通过这些事情,也可以给他们的课程安排提供一些活力,而且,参加比赛的同学也能通过他们之间的交流获得提升。

  虽然2010年11月19日,教育部宣布取消了各项奥林匹克竞赛全国决赛一等奖以下的高校保送资格,变成了由
  所在地招生委员会决定是否给予20分及以下的加分。之后又宣布2015年1月1日起,
  将取消奥赛等6项全国性鼓励类加分项目。但竞赛在学生升学其他方面还是有很大帮助的,因为如果学生在全国信息学奥林 匹克竞赛中获得一等奖,就将获得参加高校的自主招生的机会。高校会根据学生的自主招生考试结果而给予一些优惠。所以竞赛本身有助于中学生升学 的驱动力仍在。另外,对于那些对信息学或者计算机相关的知识有兴趣的同学,
  给予他们一个方便的接触这类知识的窗口也是很有必要,很有意义的一件事。

     \subsubsection{传统教育现状}
     在中学中,传统的竞赛辅导的现状是,最开始在学校中进行选拔,通过考试或者老师挑选的方式选择一些比较优秀的学生集中授课,而授课一般零散 的安排在平时日常学习的零碎时间,或者利用周末,寒暑假组成班级的集中授课,或者学生私下另外单独辅导的小班授课。
     如果集中授课,大多是很多人聚在一起,由老师统一教学,而私下授课,则需要老师学生提前安排好时间地点,
     手把手重点进行教学,而这和统一授课相比,需要的条件更多,资源也就更为稀缺。

     课堂教学分为理论部分和实际操作部分,理论会教竞赛所需编程语言的基本语法,基本只教到数组及循环,
     竞赛不会用到过于复杂的语言特性,理论教完之后,会讲一些对应知识点的习题,
     这些习题多已经成套成体系,方便系统学习。
     实际操作部分是大家集中在机房,分为自主上机练习和老师演示讲解两部分,自主上级练习是学生自己练习理论学到的语法或者做一些习题
     巩固知识,老师演示是老师边写代码,边讲解所用到的知识,学生在自己电脑上观看整个流程。

     \subsubsection{传统教育问题分析}
     从现状中,我们可以发现传统的教育方式存在很多弊端,具体列举如下:

     \begin{enumerate}
       \item 没有那么多的老师教,教师资源的稀缺,导致不是每个愿意学习信息化的学生都能公平的得到学习的机会
       \item 每个老师也没有那么多的时间和每一个学生交流,导致不是每个学生都有机会得到解惑的机会
       \item 传统的教育方式,互动模式单一,不是最适合信息学教学的方式,因为它需要更多的实际操作经验
       \item 传统的教育方式,每年老师会做太多的重复性工作,而不能解放时间,做其他竞赛相关的工作
       \item 每个学生基础不同,传统教育大班授课,不利于因材施教
       \item 传统教育由于经常是只有老师一个人在那里讲,不能给人充分思考的时间
       \item 传统小班授课,需要提前约好,时间空间都要一致,无论是金钱成本还是时间成本都很高
       \item 传统教学方式,不能很好模拟真实竞赛环境
     \end{enumerate}

     \subsubsection{OJMOOC的问题解决方案}
     解决以上发现的问题,是我们所做产品的动机,而针对问题,提出的解决方案就是利用飞速发展的互联网技术,做出一个基于web的MOOC系统。 MOOC即massive open online course,基本方式是实现在线的课程教育,老师可以把自己做好的课程及习题视频传到网络上,然后学生只要有网络就可以随时随地观看学习,
     而且不止课程,习题也可以在网上就可以得到练习,并且老师自己可以通过合理的安排自己的课程和习题顺序,来达到教学的最好效果。

     相比于传统的教育方式,OJMOOC的优势在于:
     \begin{enumerate}
       \item 老师不再需要每年做重复性工作,解放了时间,面对面的时间可以更加有效的用于解惑或者其他MOOC解决不了的问题
       \item 通过合理安排课程的前驱后继,让学生循序渐进,并对整个课程框架有一个整体的认识
       \item 通过后台的数据分析,更易发现学生易出错的地方,实现个性化教学,因材施教
       \item 任意的时间,任意想学就学的时间,任意想思考就思考的时间,任意想练习就练习的时间,而不是完全跟着老师的步子走
       \item 方便学生之间互相交流学习,也方便老师之间互相交流学习
       \item 可以重复多次听课,不用担心上课只能听一遍听不懂
       \item 老师和学生的双向选择,有利于得到最有价值的课程
     \end{enumerate}

  \subsection{同类产品比较分析}
  目前市场上已经有了一些同类型的产品,而且有些已经非常成功,有了非常成熟的运行机制。
  另外本论文重点在于前端内容及交互与设计方式,关于这方面对比分析如下:

     \subsubsection{洛谷}
     洛谷是以题目和竞赛为导向的一个网站,目标用户是要参加竞赛的个人或者团体,题目总量较多,但只是单纯的针对题目,而跟题目有关所 需知识没有关系,竞赛有多种模式,个人赛,团队赛等等,而且竞赛形式有针对NOIP,OI,ACM的多种形式,但同样不涉及知识的教育,关注 于题目的求解和训练,该网站关于题目和竞赛的展现和管理方式有一定参考价值。

     另外一些需要关注的地方是洛谷的团队申请是通过在论坛里面发帖,通过审批才能组成团队,
     是一个比较低效的方式,洛谷的题目上传是所有人都可以上传的,但各方面都有限制,
     需通过审核后会无条件成为公共题库里面的内容,否则只是私人题目,比赛的设计是参考了各个联赛的模式,分个人赛或团体赛,可以私人举办
     也可以官方举办,用户想要的参与的话,可以报名,但有的比赛还需要邀请码指定人才能参与,论坛分为站务区,题目总版,学术版,灌水区,
     这么分开来确实比混在一起有利于管理,但做的并不是特别显眼,可以思考有没有更好的方式,洛谷还有一个专门的模块的用来显示所有用户的记录
     内容包括每道题的时间内存代码长度的记录,但管理很低效,如果使用的话初看很难找到有用的信息。

     \subsubsection{慕课}
     和洛谷正好相反,慕课完全以课程为导向,目标用户是IT相关的人员,由于课程内容普遍较初级,因此更偏向于初级用户,每部分课程内容每
     个知识点都十分详尽,不但可以断点做笔记,而且从代码到结果演示,十分清晰,但是跟竞赛及题目训练没有关系,网站重点在于课程本身的学习。

     由于本项目也是以课程为导向,所以慕课关于课程的展现和管理方式有一定参考价值,课程内容从所有课程,课程推荐,到帮用户指定学习计划都有,
     对于教学来说,主题涵盖方位比较全面,而慕课在类似洛谷论坛之类所使用的展现方式是问答的形式,而且问题可以关注,这和知乎,Quora有些相似,
     为了减少零回复问题,把等待回答的问题单独列了出来,也是一个设计比较好的地方,问题种类有技术问题,课程问题,把技术分享,还有希望发展到
     线下的活动建议和他们放在了一起,可能是因为不好单独拿出来放在一个地方,最后是关于用户自己信息的展示,分为课程,计划,问答,笔记,代码,
     他们以类似标签页的形式将展示空间折叠放在一起,有把细分的内部小标题放在上部,不是把每个模块都单独放一块区域,
     设计很方便清晰,易于使用,不会因为目录项太多而有杂乱的感觉。

     \subsubsection{可汗学院}
     可汗学院和慕课一样,同样重点在于课程的学习,但区别在一是可汗学院包涵的课程范围更广泛,关于编程的部分只占一小部分,其他的类别 科目则都很多,而且课程本身的前驱后继,及过程中的流程安排更加合理,能让用户清晰意识到自己目前学习到什么程度,
     课程下方提示信息很多,介绍了学习相关用到的语法,
     但是它把所有的都展示了出来,东西太多,有些杂乱,二是它虽然已课程为主,但在课程学习的过程中掺杂了很多习题的 练习,以帮助学习,但同样目标不是竞赛,也跟算法等等相关没有关系,课程学习中,视频教学和题目练习穿插在一起,
     对于信息学重视实践来说,这点很重要,可以立刻动手巩固记忆,三是它的课程节点不只是视频,包括了文章,测试,练习等等,
     这也是由于它课程内容丰富,涉及许多学科所需要的,四是它种类多样,但每门课只有一套课程,而本项目目标是在已有课程范围的情况下,
     每门课都有多个老师去教授,学生去选择自己熟悉或者喜欢的老师。

     \subsubsection{其它}
     Coursera \& Edx:

     Coursera和Edx都是MOOC领域内非常有名的几个网站之一,它的主要方式是把名师或者名校的视频拿过来,供大家学习,各领域都有,而且都 是质量很高的课程,在Edx学成之后还会颁发证书。

     CodeVS \& CodeStudio:

     CodeVS和洛谷一样关注于题目,但比洛谷关注点更为单一,只在于题目,各类型的题目,各竞赛的题目等等,没有组织竞赛的形式。CodeStudio 关注于非常初级的用户IT方面的学习,设置任务使用图形化的方法让用户学习。


     \subsubsection{自身OJMOOC项目特点}
     OJMOOC将目标用户设定为参与NOIP竞赛的学生和老师,在课程和题目两方面都兼顾,构成在线辅导竞赛的完整流程和体系。(这里在需求进一步明 确之后再添加)


  \subsection{论文的主要工作与组织结构}
  论文的主要工作及组织结构:

  本论文的重点在于网站前端的设计与搭建。

  第一部分引言,分析项目前景,找出问题,提出解决方案,在对当前市场进行调研和发掘,通过同类产品的分析,找到自己的优势和不同点。

  第二章对所需要使用的技术进行分析。



\end{document} 